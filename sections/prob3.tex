\section{Problem 3}
\edef\sectionheader{February 16}

\noindent \textbf{Problem} \ Let $V$ be the set of vectors in $\mathbb{R}^3$ that have the form
$\begin{bmatrix}
a+2b \\
2a-b \\
-a+b
\end{bmatrix}$,
where $a,b \in \mathbb{R}$.
\\\\
    \indent (a) \ \ \ \ Prove that $V$ is a subspace of $\mathbb{R}^3$.
\\\\
    \indent (b) \ \ \ \ Find a basis for $V$.
\\\\
\noindent \textbf{Solution}
\\\\
    \indent (a) \ \ \ \ First, note that $V$ is a subset of $\R^3$, since $V$ only consists of vectors with three real components. To prove that $V$ is a subspace of $\R^3$, we must show that $V$ is
    \begin{align*}
        \text{(i) nonempty and contains the zero vector} \\
        \text{(ii) closed under scalar multiplication} \\
        \text{(iii) closed under vector addition}.
    \end{align*}
    To show (i), we let $a=0,b=0$. Then, $a$ and $b$ are in $\R$, and we see that
    $\begin{bmatrix}
        0+20 \\
        20-0 \\
        -0+0
    \end{bmatrix}$
    =
    $\begin{bmatrix}
        0 \\
        0 \\
        0
    \end{bmatrix}$
    is in $V$. \\
    To show (ii), let $a,b,c \in \R$. Let $\vec{v} = 
        \begin{bmatrix}
            a+2b \\
            2a-b \\
            -a+b
        \end{bmatrix}$. Then, clearly $\vec{v} \in V$. We want to show that $c\vec{v} \in V$. Notice,
        \begin{align*}
            c\vec{v} & = c 
                \begin{bmatrix}
                    a+2b \\
                    2a-b \\
                    -a+b
            \end{bmatrix} \\
        & = \begin{bmatrix}
                c(a+2b) \\
                c(2a-b) \\
                c(-a+b)
            \end{bmatrix} \\
        & = \begin{bmatrix}
                ac+2bc \\
                2ac-bc \\
                -ac+bc
            \end{bmatrix}.
        \end{align*}
    Now, letting $d=ac,e=bc$, we get that
    \begin{equation*}
        c\vec{v} = \begin{bmatrix}
                        d+2e \\
                        2d-e \\
                        -d+e
                    \end{bmatrix}.
    \end{equation*}
    Since $d,e \in \R$, and $c\vec{v}$ is in the form of a vector in $V$, we have that $c\vec{v} \in V$. \\
    To show (iii), let $a,b,d,e \in \R$. Then, let $\vec{v_1} =                \begin{bmatrix}
            a+2b \\
            2a-b \\
            -a+b
        \end{bmatrix}$
    and $\vec{v_2} = 
        \begin{bmatrix}
            d+2e \\
            2d-e \\
            -d+e
        \end{bmatrix}$.
    Clearly, $\vec{v_1}$ and $\vec{v_2}$ are in $V$. We want to show that $\vec{v_1} + \vec{v_2} \in V$. Notice,
    \begin{align*}
        \vec{v_1} + \vec{v_2} & =
            \begin{bmatrix}
                a+2b \\
                2a-b \\
                -a+b
            \end{bmatrix} + 
            \begin{bmatrix}
                d+2e \\
                2d-e \\
                -d+e
        \end{bmatrix} \\
        & = \begin{bmatrix}
                a+2b+d+2e \\
                2a-b+2d-e \\
                -a+b-d+e
            \end{bmatrix} \\
        & = \begin{bmatrix}
                a+d+2(b+e) \\
                2(a+d)-(b+e) \\
                -(a+d)+b+e
            \end{bmatrix}.
    \end{align*}
    Let $f=a+d,g=b+e$. Then, $f,g \in \R$, and
    \begin{equation*}
        \vec{v_1}+\vec{v_2} =
            \begin{bmatrix}
                f+2g \\
                2f-g \\
                -f+g
            \end{bmatrix}.
    \end{equation*}
    This is clearly in the form of a vector in $V$. Hence, $\vec{v_1} + \vec{v_2} \in V$. \\
    Since $V \subseteq \R^3$, and since (i), (ii), and (iii) hold, we conclude that $V$ is a subspace of $\R^3$.
\\\\
    \indent (b) \ \ \ \ To find a basis for $V$, we must find a set of linearly independent vectors that span $V$. If we let $a=1,b=0$, then we get the vector 
        $\begin{bmatrix}
            1 \\
            2 \\
            -1
        \end{bmatrix}$.
    Now if we let $a=0,b=1$, we get
        $\begin{bmatrix}
            2 \\
            -1 \\
            1
        \end{bmatrix}$.
    Since these are not scalar multiples of each other, they are linearly independent. Now, notice that any vector in $V$ is of the form
    $\begin{bmatrix}
        a+2b \\
        2a-b \\
        -a+b
    \end{bmatrix} =$
    $\begin{bmatrix}
        a \\
        2a \\
        -a
    \end{bmatrix} + $
    $\begin{bmatrix}
        2b \\
        b \\
        b
    \end{bmatrix}$,
    with $a,b \in \R$. This is a linear combination of two independent vectors in $\R^3$. Therefore, $V$ is a plane in $\R^3$, and a possible basis for $V$ is
    $\left\{ \begin{bmatrix}
            1 \\
            2 \\
            -1
            \end{bmatrix},
            \begin{bmatrix}
                2 \\
                -1 \\
                1
            \end{bmatrix} \right\}$.